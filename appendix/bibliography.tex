\BiAppChapter{参考书目}{}
这份书目包括了系统编程相关的推荐读物,分为四个部分进行介绍。阅读本书时并不需要读这些著作。它们是我认为在相关主题上最好的书。如果你希望深入了解某些主题,我强烈推荐这些书。

部分书籍所讨论的内容是假设本书的读者已经熟悉的(例如C语言)。某些则是对本书的有效补充,例如介绍gdb,Subversion(svn),以及操作系统设计方面的书籍。另外一些所讨论的问题超出了本书的范围(例如套接字的多线程编程)。无论其内容如何,我向大家推荐所有的书。当然,这份书单算不上详尽——你可以自由选择其它读物。

\BiSection{C语言程序设计的相关书籍}{}
以下书籍介绍了系统编程的通用语言--- C语言。如果你不能熟练地编写C语言代码,以下的书籍(辅以大量的练习!)应该能在这方面提供帮助。至少,第一本——以K\&R而广为人知——是非常适合阅读的。它简短的篇幅很好的体现了C语言的简单性。

\textit{The C Programming Language}, 2nd ed. Brian W. Kernighan and Dennis M. Ritchie.
    Prentice Hall, 1988.
    这本书由C程序设计语言的作者和他的伙伴所著,被称作C语言圣经。

\textit{C in a Nutshell}. Peter Prinz and Tony Crawford. O’Reilly Media, 2005.
    一本很好的关于C语言和C标准库的书籍。
\textit{C Pocket Reference}. Peter Prinz and Ulla Kirch-Prinz. Translated by Tony Crawford.
    O’Reilly Media, 2002.
    一份简明扼要的C语言参考,已经更新到最新的ANSI C99标准。
\textit{Expert C Programming}. Peter van der Linden. Prentice Hall, 1994.
    该书对C语言中较少为人所知的部分进行了一次精彩讨论,行文中闪现着作者令人惊奇的才智和幽默感。该书充满了无厘头的笑话,不过我喜欢。
\textit{C Programming FAQs: Frequently Asked Questions}, 2nd ed. Steve Summit. Addison-Wesley, 1995.
    
    这本大部头囊括了超过400个C程序设计语言的常见问题(包括答案)。许多FAQ在C语言专家眼中是小菜一碟,但对于一些重要问题的问答,即使是最博学的C程序员都会被雷到。如果你能解决所有这些“怪物”,你绝对是一个C语言忍者!该书唯一不足就没有跟上ANSI C99,而这肯定会带来一些变化(我在自己手头的书中已经做了修正)。需要注意的是,有一份在线版本的貌似已经做了更新。

\BiSection{Linux编程的相关书籍}{}
下面推荐的书籍主要介绍Linux编程的相关主题,其中包括了本书没有讨论的主题(套接字,IPC,以及pthreads),和Linux编程工具(CVS,GNU Make,以及Subversion)。

\textit{Unix Network Programming, Volume 1: The Sockets Networking API}, 3rd ed. W. Rich-
    ard Stevens et al. Addison-Wesley, 2003.
    套接字API的绝对巨著;可惜并不是针对Linux的,不过最近更新到了IPv6。

\textit{UNIX Network Programming, Volume 2: Interprocess Communications}, 2nd ed.
   W. Richard Stevens. Prentice Hall, 1998.
    关于进程间通信(IPC)的绝佳讨论。

\textit{PThreads Programming: A POSIX Standard for Better Multiprocessing}. Bradford
   Nichols et al. O’Reilly Media, 1996.
    关于POSIX线程API---pthreads的评述。

\textit{Managing Projects with GNU Make}, 3rd ed. Robert Mecklenburg. O’Reilly Media,
   2004.
    关于GNU Make---Linux上建立软件项目的经典工具的绝佳描述。

\textit{Essential CVS}, 2nd ed. Jennifer Versperman. O’Reilly Media, 2006.
    关于CVS---Unix系统上版本控制和源码管理的经典工具的绝佳描述。

\textit{Version Control with Subversion}. Ben Collins-Sussman et al. O’Reilly Media, 2004.
    关于Subversion---Unix系统上版本控制和源码管理的优秀工具的令人惊奇的叙述,由Subversion的三位作者完成。

\textit{GDB Pocket Reference}. Arnold Robbins. O’Reilly Media, 2005.
    一份gdb---Linux调试器的袖珍指南。

\textit{Linux in a Nutshell}, 5th ed. Ellen Siever et al. O’Reilly Media, 2005.
    对Linux中各种内容的快速浏览,包括许多Linux开发环境下的工具。

\BiSection{Linux内核的相关书籍}{}
下面列出的两本书主要涉及Linux内核方面的主题。我们有三重理由来对该主题进行研究。首先,内核提供了对用户空间的系统调用接口。其次,内核的行为和特性会在与其上运行的应用进行交互时体现出来。最后,Linux内核代码非常优美,同时这些书很有意思。

\textit{Linux Kernel Development}, 2nd ed. Robert Love. Novell Press, 2005.
    该书非常适合给那些希望了解Linux内核设计实现的系统程序员阅读(很显然,我就不必提及我在该主题上的看法了)。该书不仅可作为API参考,同时也对Linux内核中使用的算法以及所做的决策也做了精彩的论述。
\textit{Linux Device Drivers}, 3rd ed. Jonathan Corbet et al. O’Reilly Media, 2005.
    本书是在编写Linux内核设备驱动程序方面的绝佳指南,同时也是非常棒的API参考手册。尽管针对的是设备驱动,但书中的讨论可以使各类程序员受益(包括很少探究Linux内核机制的系统程序员)。该书是我在Linux内核方面的必备书籍。

\BiSection{操作系统设计的相关书籍}{}
这两本书不是针对Linux的,而是从理论上介绍操作系统设计与实现。正如我在本书所强调的那样,对你进行编程其上的系统有一个良好的认识颇有助益。

\textit{Operating Systems}, 3rd ed. Harvey Deitel et al. Prentice Hall, 2003.
    操作系统设计理论方面的力作,同时还包括将理论付诸实践的顶尖样例分析。在所有操作系统设计书籍中,这是我的最爱:它紧随操作系统的研究发展,易读且详尽。

\textit{UNIX Systems for Modern Architectures: Symmetric Multiprocessing and Caching for Kernel Programming}. Curt Schimmel. Addison-Wesley, 1994.
   
    尽管和系统编程的关系不大,但该书为并发的危险性和现代缓存系统提供了绝佳的描述。吐血推荐!



