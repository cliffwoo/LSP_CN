%\newcommand{\chuhao}{\fontsize{42pt}{\baselineskip}\selectfont}
%\newcommand{\xiaochuhao}{\fontsize{36pt}{\baselineskip}\selectfont}
%\newcommand{\xiaoerhao}{\fontsize{18pt}{\baselineskip}\selectfont}
%\newcommand{\xiaosihao}{\fontsize{12pt}{\baselineskip}\selectfont}
%\newcommand{\xiaowuhao}{\fontsize{9pt}{\baselineskip}\selectfont}
\vspace*{16ex}
\begin{center}
	\textit{\huge 谨将我们的工作献给}
\end{center}

\begin{center}
	\textit{\Large 即将毕业离校的兄弟们} --- \textit{\Large 林晓鑫、刘德超、黄巍、周蓝珺、胡禹轩、王新喜、何春晓、崔剑、李浩。}
\end{center}

\begin{center}
	\textit{\large 以及}
\end{center}

\begin{center}
	\textit{\Large 潘海东即将出世的小Baby!}
\end{center}

\newpage
\chapter{译者序}
《Linux System Programming》(简称LSP)的中文翻译工作是浮图开放实验室和哈尔滨工业大学计算机学院IBM俱乐部 《深入理解计算机系统》讨论班的练习项目。参与翻译工作的同学包括从本科二年级到研究生二年级的十一位同学。他们是林晓鑫、王澍、崔玉春、吉飞飞、何春晓、熊飞、李志、张祖羽、张智、陈盛、张永辉。附录的翻译由SMS@lilacbbs.com完成,参考文献部分的翻译由王澍完成。刘文懋、王耀、刘德超、于墨、王新喜等同学参与了审校工作。全书的初稿审校由吴晋完成。本书基于哈尔滨工业大学硕士博士论文TeX模板制作,全书的模板修正工作由李志完成。

本书的翻译工作基于LSP第一版完成,并根据英文版勘误进行了修正。

在本书内部审核版本发布后得到了原IBM俱乐部成员谢煜波(现供职于微软亚洲工程院)和戴晓光(现供职于SUN中国有限公司)的大力支持,他们利用宝贵的业余时间对部分章节进行了仔细的审校,并提出了大量细致的修改意见。他们的修改意见使所有参与翻译的同学受益良多。在此向他们表示感谢。

在翻译过程中还得到了紫丁香社区和Harbin Linux User Group网友的大力支持,在此向他们一并表示感谢。

由于译者在系统编程方面并没有丰富的经验,整个翻译工作以学习为目的,因此书中的错误和疏漏在所难免。如果书中存在任何问题,请用如下方式和我们联系:
\begin{center}
\begin{description}
%	\item[Website]:
%		\begin{enumerate}
	\item[Website]: http://www.footoo.org
%		\item http://sites.google.com/a/footoo.org/footoo-lab/
%		\end{enumerate}
	\item[Twitter]: http://twitter.com/cliffwoo
	\item[Email]:\mbox{cliffwoo@gmail.com} 或者 \mbox{cliffwoo@footoo.org}
	\item[Google Groups]:http://groups.google.com/group/lspcn/
\end{description}
\end{center}
\begin{flushright}
	吴晋 于 哈尔滨工业大学\linebreak[2]
	
	2009年4月30日
\end{flushright}
