\ifx\atempxetex\usewhat
\XeTeXinputencoding "utf-8"
\fi
\defaultfont

\chapter{高级进程管理}

在第五章,我们介绍了进程的基本概念,讨论了创建、控制和销毁进程的内核接口。基于这些知识,本章将首先讨论Linux进程调度器及其调度算法,然后描述高级进程管理接口。这些系统调用控制进程的调度语义和行为,从而影响调度器为实现应用或者用户特定需求所作出的决策。

\section{进程调度}

进程调度器是内核中决定哪个进程可以运行的组件,换句话说,进程调度器简称------调度器------是把有限的处理器资源分配给进程的内核子系统。在作出决策的过程中,调度器既要最大化处理器效率,又要让多个进程同时运行、互不影响。

本章,我们会着重讨论“就绪进程”\footnote[1]{译者注:原文 runnable process}。它最重要的特征是该进程是非阻塞的。进行用户交互、大量读写文件、响应I/O和网络事件的进程会花费大量时间来等待资源可用,在相当长的时间内无法转为就绪状态(长是相对于指令运行时间而言),因此就绪进程首先应该是非阻塞的。一个就绪进程还必须至少有部分“时间片”(调度器分配给进程的运行时间)。内核用一个就绪队列维护所有的就绪进程,一旦某进程耗光它的时间片,内核就将其移出队列,直到所有就绪进程都耗光时间片才考虑将其放回队列。

如果只有一个就绪进程(甚至一个也没有),调度器就没有什么意义。只有在进程数多于处理器时,调度器才能体现它的价值。在这种情况下,显然有一些进程会等待其它进程运行;决定哪个进程运行,何时运行,运行多长时间也就成了调度器的基本责任。

一个操作系统能在单处理机上交错地运行多个进程,让人感到似乎同时运行多个进程,就称该操作系统是“多任务”的。在多处理机上,多任务操作系统允许进程在不同处理器上并行执行。非多任务操作系统,比如DOS,一次只能运行一个任务。

多任务操作系统分为两大类:协同式和抢占式。Linux实现了后一种形式的多任务,调度器可以要求一个进程停止运行,处理器转而运行另一个进程。这种中止正在运行的进程的行为称做抢占,类似的,进程在被抢占前所运行的时间称之为进程时间片(得名于调度器分配给每个就绪进程的一小片时间)。

在协同多任务系统中,一个进程持续运行直到它自发停止。我们称进程自发停止的行为为让出。理想情况下,会经常发生进程让出,但操作系统绝不可强制要求其让出。因此,一个拙劣或损坏的程序可能运行很长时间,甚至导致整个系统死掉。由于这个原因,现代操作系统几乎都采用抢占多任务机制,Linux也不例外。

2.5内核引入的O(1)进程调度器,是Linux调度\footnote[1]{对于好奇的读者,可以查看内核代码树中的 kernel/sched.c 文件了解细节}的核心。Linux调度算法采用抢占多任务机制,支持多处理器,处理器亲和度,非一致内存访问(NUMA),实时进程和用户自定义优先级等特性。

\subsection{大O记法}
O(1)是大O记法的一个例子,常用来表示算法的复杂性和可扩展性。形式地定义:
\begin{center}
  if f(x) is O(g(x)),

  then

  $\exists c,x'$ such that $f(x) \le c * g(x), \forall x > x'$
\end{center}

也就是说,一些算法的代价,以f表示,只要x大于某个初始值$x'$,f就一定小于等于g乘上一个常数,即g大于等于f,g是f的上界。

O(1)也就表明解决问题的代价小于常数c,这就带来极好的保证:无论系统中有多少进程,Linux进程调度器表现如一。对于调度器来说,选择一个新进程来运行至少需要迭代一次进程队列,因而O(1)算法对性能非常重要。在稍简单些的调度器中(包括以前版本的Linux调度器),随着进程数量的增长,进程队列的迭代会成为潜在的瓶颈。在最好情况下,也会给进程调度带来不确定性。

Linux调度器,在常数时间内运行,不会受到其它因素影响,就没有这样的瓶颈。

\subsection{时间片}

Linux分配给进程的时间片长短对于系统行为和性能非常重要。如果时间片太长,进程在执行前必须等待很长时间,这降低了表现出来的运行并行性,用户会察觉到明显的延迟;相反的,时间片太短,大量时间会花费在进程调度上,程序的时间局部性等也不能得到保证。

因而,确定合适的时间片绝非易事。一些系统给予进程长时间片,希望最大化系统吞吐率和整体表现。另一些系统给予较短的时间片,希望获得优秀的交互性能。我们将看到,Linux通过动态分配进程时间片,期望在两方面都做到最好。

注意:进程不一定要在一次运行中耗光所有时间片。一个被分配100ms时间片的进程,可能运行20ms就因为等待键盘输入等资源而阻塞。此时,调度器就会临时地把该进程移出就绪队列;当资源可用后,这个例子中是键盘缓冲区不为空,调度器会唤醒进程。进程会继续运行,耗光剩下的80ms或者又一次阻塞。

\subsection{I/O约束进程 Vs. 处理器约束进程}

持续地消耗所有可用时间片的进程称为“处理器约束进程”\footnote[1]{process-bound}。这类进程渴望CPU时间,消耗掉调度器分配的全部时间。最简单的例子就是无限循环,其它的例子包括科学计算,数学演算和图像处理。

另一方面,多数时间处于等待资源的阻塞状态的进程称为“I/O约束进程”\footnote[2]{I/O-bound}。I/O约束进程经常发起和等待文件I/O,阻塞在键盘输入,或者用户移动鼠标。I/O阻塞程序的例子包括文件实用程序,比如 cp 或者 mv,它们除了请求内核执行I/O操作外,几乎什么也不做;还包括GUI应用程序,大多数时候都在等待用户输入。

处理器约束程序和I/O约束程序得益于调度器对于不同程序类型所采用的不同行为。处理器约束程序得到尽可能长的时间片,从而最大化缓存命中率(时间局部性所致),尽快完成任务。相反地,I/O约束程序不需要长时间片,因为它们一般在发出I/O请求和阻塞在内核资源前只会运行很短的一段时间。然而,I/O约束程序也能从调度器的持久关注中获益。被唤醒地越快,可调度的I/O请求越多,程序越能够充分地利用系统硬件。更进一步,如果一个在等待用户输入的程序被调度的速度越快,给用户无缝运行的感觉越明显。

平衡处理器约束程序和I/O约束程序的不同需要绝非易事。Linux调度器试图识别和优先对待I/O约束程序:频繁I/O程序增加优先级,处理器约束程序减少优先级。

实际上,很多应用程序是处理器约束和I/O约束的混合体。音频视频解码编码就是一个好例子,许多游戏也是如此。因此对于一个给定程序,不总是能够判定它的倾向,在任意时间,进程都可能有不同的表现。

\subsection{抢占调度}

当一个进程耗光时间片的时候,调度器会中止其运行,开始运行一个新的进程。如果没有其它的就绪进程,内核会给予所有耗光时间片的进程新的时间片,继续运行。这样,即使系统中有高优先级进程---低优先级进程必须等待高优先级进程耗光时间片或者阻塞,所有的进程最终都会运行。这就明确描述了Unix调度中一条不起眼但是非常重要的原则: 所有的进程都必须运行。

如果没有就绪进程,内核会“运行”空闲进程(idle process)。实际上,空闲进程既不是一个进程,也不能实际运行(减少电池消耗)。空闲进程是为了简化调度算法和统计方便而生的特殊例程,空闲时间简化为运行空闲进程的时间。

在进程运行时,如果另一个高优先级进程就绪(也许先前此进程阻塞在键盘输入,用户正好敲入一个单词),当前运行进程被直接中止,切换到高优先级进程。因此,不会有就绪却没有运行的较高优先级进程,系统中的运行进程一定是最高优先级的可运行进程。

\subsection{线程}

线程是进程中的运行单元,所有的进程都至少有一个线程。每一个线程都独自占有一个虚拟处理器:独自的寄存器组,指令计数器和处理器状态。虽然多数进程都只有一个线程,但是进程实际可以拥有很多线程,每个线程完成不同的任务,但是共享同一地址空间(也就是同样的动态内存,映射文件,目标代码等等),打开的文件队列和其它内核资源。

Linux内核对于线程的观点独特而有趣。本质上,内核没有线程概念,对于Linux内核来说,所有的线程都是独立的进程。广义上说,两个无关进程和一个进程内的两个线程没有区别。内核把线程简化为共享资源的进程,也就是说,内核把一个进程中的两个线程,简化为共享一系列内核资源(地址空间,打开的文件列表等)的两个不同进程。

多线程编程是基于线程模型的编程技术。Linux平台上线程编程的最常用API是由IEEE Std 1003.1c-1995(POSIX 1995 or POSIX.1c)所标准化的API,开发者们称实现此API的库为“pthreads”。线程编程是一个复杂的课题,相关API也很艰深复杂。因而,pthreads不在本书的讨论范围内,相反,本书把注意力放在建构pthreads库的那些接口上。

\section{让出处理器}

虽然Linux是一个抢占多任务操作系统,但是它也提供了一个系统调用来允许进程主动让出处理器,并通知调度器选择另一个进程来运行。
% \setmonofont{DejaVu Sans Mono}
\begin{lstlisting}
  #include <sched.h>
  int sched_yield (void);
\end{lstlisting}

调用 sched\_yield() 函数将中断当前进程,运行一个新进程,就和内核主动抢占进程一样。注意,在多数情况下,系统中并没有其它就绪进程,让出的进程会直接恢复运行。由于这种不确定性,加上人们认为可以做更好的选择,这一系统调用并不经常使用。

调用成功返回0,失败返回-1,并设置“errno”。在包括Linux在内的多数Unix系统上,sched\_yield()不可能失败,因此总是返回0。然而,一个严谨的程序员还是会检查返回值:

\begin{lstlisting}
  if(sched_yield ())
    perror ("sched_yield");
\end{lstlisting}

\subsection{合理使用}

在Linux系统这样的抢占多任务系统中,很少有合理使用sched\_yield()的机会。内核完全有能力作出最优化和最有效率的调度决策,这是因为,内核显然比一个独立的应用程序更有资格决定何时抢占哪个进程,协同多任务和抢占多任务两种不同机制来源于对此的不同理解。

那么为什么会有这么一个系统调用呢?这取决于应用程序是否必须等待用户,硬件或者其它进程所触发的外部事件。比如说,如果一个进程必须等待另一个进程,“让出处理器,直到另一个进程完成”,这是一个直观的想法。以一个消费者/生产者模型中简单的消费者实现为例,大概是这样:

\begin{lstlisting}
  /* the consumer... */
  do {
    while (producer_not_ready ())
    sched_yield ();
    process_data ();
  } while (!time_to_quit ());
\end{lstlisting}

幸运的是,Unix程序员一般不需要这样编写代码。Unix程序通常是事件驱动的,倾向于利用阻塞机制(比如管道)来代替sched\_yield()。这种情况下,消费者从管道读取数据,在必要的时候阻塞等待数据。这就使用户空间进程摆脱了进程协同的责任,交给了内核;而对于内核来说,又可以用进程睡眠的方式来优化管理,并在需要的时候激活。一般来说,Unix程序倾向于使用建立在可阻塞文件描述符基础上的事件驱动机制。

直到最近,有一种情况需要讨厌的sched\_yield():用户空间线程锁。当一个线程试图请求另一个线程已经拥有的锁的时候,该线程会让出处理器直到锁可用。在内核不支持用户空间锁的时候,这种方法最简单高效。然而,现代Linux线程实现(the New POSIX Threading Library, or NPTL)迎来了一个基于快速用户互斥锁的优化方案,即在内核中提供用户空间锁的支持。

sched\_yield()的另一个应用是“变得更友好”(playing nicely): 一个处理器密集的程序可以不时调用sched\_yield()来减少对系统的影响。出发点不错,但是也有两个缺点。第一,内核能比一个独立进程作出更好的全局调度决策,因此,平缓系统操作的责任应该由调度器承担,而不是用户进程。对此,调度器也恰好有奖励I/O密集程序,惩罚处理器密集程序的策略。第二,减轻处理器密集程序带来的负担,从而保证其它程序的运行,这是用户的责任,而不是应用程序。用户可以通过后面将讨论的``nice"命令来给应用程序传递用户关于性能的设置。

\subsection{让出处理器方法的过去和现状}

在2.6内核以前,调用sched\_yield()产生一个很简单的作用。如果有其它就绪进程,内核会运行该进程,并把当前进程放到就绪队列的末尾,短期内就会再次调度到此进程。当然也很有可能没有其它进程,那么当前进程继续运行。

2.6内核做了调整,算法如下:

\begin{enumerate}
\item 进程是实时进程吗?如果是,将其放到就绪队列的末尾,返回(和以前一样)。如果不是,到下一步。(想了解实时进程,请阅读本章“实时系统”一节。)
\item 把该进程从就绪队列移出,放到到期进程队列中。也就是说,只有当所有就绪进程运行,耗光时间片后,该进程才能和所有的到期进程一道重新进入就绪队列。
\item 调度下一个就绪进程运行。
\end{enumerate}

因此,调用sched\_yield()的实际作用就和进程耗光时间片一样,这不同于早期内核的处理,那时sched\_yield()的效果很轻微(等同于“如果有一个进程就绪等待,运行它,然后回到我”)。

改变的原因之一是为了预防称之为“乒乓”的不合理情况发生。想象两个进程 A 和 B ,都调用了sched\_yield()。 假设它们俩是仅有的两个就绪进程(也可以有其它进程,但这些进程没有时间片)。如果是以前的sched\_yield(),内核都会轮流调度两个进程,每个进程都请求执行其它进程,直到两个进程的时间片都用光。我们可以画一个图表来描述内核的调度选择,大概是``A, B, A, B, A, B"等等,因此得名“乒乓”。

2.6内核就不会发生这种情况。当A请求让出处理器的时候, 调度器就会将它移出就绪队列。 类似地,B也会被移出。当没有就绪进程的时候,调度器当然不会考虑运行A 还是B, 也就防止了乒乓效应,允许其它进程获得处理器时间。

因此,当进程请求让出处理器的时候,它就应该是真的要让出处理器!

\section{进程优先级}

\begin{wrapfigure}{l}{2.5cm}
  \includegraphics[width=2cm,clip]{paipai.png}
\end{wrapfigure}
\mbox{}\begin{flushleft}本节讨论常用的非实时进程,实时进程需要不同的调度标准和独立的优先级系统,留到以后讨论。\end{flushleft}

Linux不是随意地进行进程调度。进程都有一个影响它们何时运行,运行多久的优先级。历史上,Unix把这个优先级称为“nice values”,因为低优先级的进程往往允许其它进程分享更多的处理器时间,也就意味着该进程对系统的其它进程更加友好。

``nice value"是在进程运行的时候指定的,Linux调度器基于这样的原则来调度:高优先级的程序总是先运行。同时,nice值也指明了进程的时间片长度。

合法的优先级在-20到19之间,默认为0。有些让人疑惑的是,nice值越低,优先级越高,时间片越长;相反,nice值越高,优先级越低,时间片越短,增加一个进程的nice值意味着该进程对系统更友好。数字令人容易混淆。当我们说一个进程有高优先级的时候,我们认为该进程应该更快地开始运行,运行更长的时间,显然对系统的友好度更低。

\subsection{nice()}

Linux提供了获取和设置进程nice值的系统调用,最简单的就是nice():

\begin{lstlisting}
  #include <unistd.h>
  int nice (int inc);
\end{lstlisting}

成功调用nice()将在现有优先级上增加inc,并返回新值。只有拥有CAP\_SYS\_NICE能力(实际上,就是root所有的进程)才能够使用负值inc,减少友好度,增加优先级。因此,非root进程只能降低优先级(增加nice值)。

遇到错误,nice()返回-1,但是-1也可能是成功时的返回值,因此为了区别成功与否,在调用前应该对errno置0,调用后检查。举例来说:

\begin{lstlisting}
  int ret;
  errno = 0;
  ret = nice (10);  /* increase our nice by 10 */
  if (ret == -1 && errno !=0)
    perror ("nice");
  else
    printf ("nice value is now %d\n", ret);
\end{lstlisting}

对于nice(),Linux只会返回一种错误号: EPERM,表明进程试图提高优先级而又没有CAP\_SYS\_NICE能力。其它的系统还会在nice值超出范围的时候返回 EINVAL,而Linux会把非法的值放到对应的边界上。

传递参数0给nice()是获得当前优先级的简单方法:

\begin{lstlisting}
  printf("nice value is currently %d\n", nice (0));
\end{lstlisting}

通常,进程需要设定绝对的优先级而不是相对增量的时候,可以用下边的代码:

\begin{lstlisting}
  int ret, val;
  /* get current nice value */
  val = nice (0);
  /* we want a nice value of 10 */
  val = 10 - val;
  errno = 0;
  ret = nice (val);
  if (ret == -1 && errno != 0)
    perror ("nice");
  else
    printf ("nice value is now %d\n", ret);
\end{lstlisting}

\subsection{getpriority()和setpriority()}

更好的解决方案是用getpriority()和setpriority()系统调用,可以带来更多的控制能力,当然使用起来也更为复杂:

\begin{lstlisting}
  #include <sys/time.h>
  #include <sys/resource.h>
  int getpriority (int which, int who);
  int setpriority (int which, int who, int prio);
\end{lstlisting}

两个函数作用于``which"和``who"指定的进程,进程组或者用户,其中“which“的取值为 PRIO\_PROCESS、 PRIO\_PGRP 或者 PRIO\_USER,对应地,“who”就说明了进程ID,进程组ID或者用户ID。当“who”是0的时候,分别是当前进程,当前进程组或者当前用户。

getpriority()返回指定进程中的最高优先级(nice值最小),setpriority()则将所有进程的优先级都设为“prio“。同nice()一样,只有拥有CAP\_SYS\_NICE能力的进程能够提高一个进程的优先级(降低nice值),更进一步地说,只有这样的进程才能够调整不属于当前用户的进程的优先级。

getpriority()遇到错误的时候返回-1,但-1也可能是成功的返回值,因此同nice()一样,为了处理错误情况,程序员必须在调用前清空error值。setpriority()就不必如此,它总是成功返回0,错误返回-1。

下面是得到当前进程优先级的例子:

\begin{lstlisting}
  int ret;
  ret = getpriority (PRIO_PROCESS, 0);
  printf ("nice value is %d\n", ret);
\end{lstlisting}

设置当前进程组所有进程优先级为10的例子:

\begin{lstlisting}
  int ret;
  ret = setpriority (PGIO_PGRP, 0, 10);
  if (ret == -1)
    perror ("setpriority");
\end{lstlisting}

错误的时候,函数会设置errno为以下几个值:

\begin{eqlist*}
\item[EACCESS] 进程没有CAP\_SYS\_NICE能力,却企图提高进程优先级。(仅适用于setpriority())
\item[EINVAL] ``which"的值不在PRIO\_PROCESS, PRIO\_PGRP或PRIO\_USER之中。
\item[EPERM] 指定的进程有效用户ID和调用进程有效用户ID不一致,且调用进程没有CAP\_SYS\_NICE能力。(仅适用于setpriority())
\item[ESRCH] 不存在符合“which”和“who”的进程。
\end{eqlist*}

\subsection{I/O优先级}

作为进程优先级的补充,Linux还允许进程指定I/O优先级,内核I/O调度器(参考第四章)总是优先响应来自于高I/O优先级的请求。

缺省情况下,I/O调度器用进程友好度决定I/O优先级,因此,设置优先级自动改变I/O优先级。然而,Linux内核还有两个系统调用来单独获取和设置I/O:

\begin{lstlisting}
  int ioprio_get (int which, int who)
  int ioprio_set (int which, int who, int ioprio)
\end{lstlisting}

不幸地是,内核没有导出这两个系统调用,glibc也没有提供用户空间接口。没有glibc的支持,函数用起来是相当麻烦的。以后,如果未来glibc开始支持的时候,接口有可能和系统调用不同。在这之前,有两个可移植方法来操作进程I/O优先级:通过nice值,或者类似于``ionice"\footnote[1]{ionice是util-linux包的一部分,可以从http://www.kernel.org/pub/linux/utils/utl-linux得到,以GNU General Public License v2授权}的实用程序。

不是所有的I/O调度器都支持I/O优先级,特别地,Complete Fair Queuing(CFQ) I/O调度器支持,其它的标准调度器不支持。如果I/O调度器不支持I/O优先级,相关系统调用被忽略而没有任何提示信息。

\section{处理器亲和度}

Linux可以在一个系统中使用多处理器,除了启动进程,支持多处理器的大多数工作都依赖于进程调度器。在对称多处理机(SMP)上,进程调度器必须决定每个CPU上运行哪个进程,因此,必须解决两个问题:调度器必须充分利用系统的处理器,尽量避免处理器空闲。然而,如果一个进程曾在某一CPU上运行,进程调度器还应该尽量把它放在同一CPU上,因为处理器间的进程迁移会带来性能损失。

最大的性能损失来自于迁移带来的缓存效应\footnote[2]{译者注:cache effects}。现代SMP系统的设计中,每个处理器的缓存是各自独立的,也就是说,处理器并不共享缓存中的数据。因此,当进程迁移到新处理器上后写入新数据到内存时,原有处理器的缓存就过期了,这可能带来冲突。为了避免这种情况,缓存读入新的一块内存数据时标记其它缓存无效。因此,在任意时刻,任意数据仅在一个处理器的缓存中有效(假设该数据被缓存)。当进程迁移的时候,就有两方面的相关损失:进程不再能访问缓存数据且原有缓存中的数据必须标记为无效。考虑到这些损失,进程调度器应该尽量让进程停留在一个处理器。

实际上,进程调度器的两个目标有潜在的冲突。如果一个处理器比其它处理器有大得多的进程负载---或者更糟一些,一个处理器繁忙,其它处理器空闲---重新调度进程到低负载CPU上就是有意义的。决定何时移动进程来避免不平衡,称为负载均衡,对SMP机器的性能至关重要。

处理器亲和度表明一个进程停留在同一处理器上的可能性。术语“软亲和度”(soft affinity)表明调度器持续调度进程到同一处理器上的自然倾向,从上文的讨论可以看到,这是非常有价值的特性。Linux调度器尽可能地这样做,只有当负载极端不平衡的时候,才考虑迁移进程;从而,最小化迁移的缓存效应,还能保证系统中的处理器负载基本平衡。

然而有些时候,用户或者应用程序需要保证进程和处理器间的绑定,这通常发生在进程非常依赖缓存\footnote[3]{译者注: cache-sensitive},期望停留在同一处理器的情况下。术语“硬亲和度”(hard affinity)描述了强制内核保证进程到处理器的绑定。

\subsection{sched\_getaffinity() 和 sched\_setaffinity()}

进程从父进程继承处理器亲和度;在默认情况下,可能运行在任何CPU上。Linux提供两个系统调用来获取和设定进程的硬亲和度\footnote[3]{译者注: hard affinity}:

\begin{lstlisting}
  #define _GNU_SOURCE
  #include <sched.h>
  typedef struct cpu_set_t;
  size_t CPU_SETSIZE;
  void CPU_SET (unsigned long cpu, cpu_set_t *set);
  void CPU_CLR (unsigned long cpu, cpu_set_t *set);
  int CPU_ISSET (unsigned long cpu, cpu_set_t *set);
  void CPU_ZERO (cpu_set_t *set);
  int sched_setaffinity (pid_t pid, size_t setsize, const cpu_set_t *set);
  int sched_getaffinity (pid_t pid, size_t setsize, const cpu_set_t *set);
\end{lstlisting}

调用sched\_getaffinity()可以获得由``pid"指定的进程的处理器亲和度,存储在cpu\_set\_t类型中,可以用特殊的宏访问。如果“pid”是0,则得到当前进程的亲和度。“setsize”参数是cpu\_set\_t的大小,glibc用它来保证将来类型变化时依然具有兼容性。成功的时候,函数返回0;错误返回-1,并设置errno。例子如下:

\begin{lstlisting}
  cpu_set_t set;
  int ret, i;
  CPU_ZERO (&set);
  ret = sched_getaffinity (0, sizeof (cpu_set_t), &set);
  if (ret == -1)
  perror ("sched_getaffinity");
  for (i = 0; i< CPU_SETSIZE; i++) {
    int cpu;
    cpu = CPU_ISSET (i, &set);
    printf ("cpu=%i is %s\n", i, cpu ? "set" : "unset" );
  }
\end{lstlisting}

在调用前,我们用CPU\_ZERO清零所有的二进制位,然后开始从0到CPU\_SETSIZE在set上的迭代。注意,CPU\_SETSIZE并不是set的大小---显然不能用它表示setsize---而是set可能表示的处理器数量。因为现在的实现用1个二进制位表示一个处理器,所以CPU\_SETSIZE比sizeof(cpu\_set\_t)大得多。我们用CPU\_ISSET检查系统中的处理器是否被绑定到这个进程,0表示否,非0则已经绑定。

只有实际存在的处理器才能被绑定,因此,在两个处理器的系统上运行上述代码将得到如下结果:

\begin{verbatim}
  cpu=0 is set
  cpu=1 is set
  cpu=2 is unset
  cpu=3 is unset
  ...
  cpu=1023 is unset
\end{verbatim}

从输出中可以看到,当前的CPU\_SETSIZE(从0开始)是1,024。

我们考虑CPU \#0和\#1这两个仅存的处理器,也许我们期望保证我们的进程仅仅运行在CPU \#0上。代码如下:

\begin{lstlisting}
  cpu_set_t set;
  int ret, i;
  CPU_ZERO (&set);        /* clear all CPUs */
  CPU_SET (0, &set);      /* allow CPU #0 */
  CPU_CLR (1, &set);      /* forbid CPU #1 */
  ret = sched_setaffinity (0, sizeof (cpu_set_t), &set);
  if (ret == -1)
    perror ("sched_setaffinity");
  for (i = 0; i < CPU_SETSIZE; i++) {
    int cpu;
    cpu = CPU_ISSET (i, &set);
    printf ("cpu=%i is %s\n", i, cpu ? "set" : "unset");
  }
\end{lstlisting}

  我们首先总是用CPU\_ZERO清零set,然后用CPU\_SET对CPU \#0置1, 用CPU\_CLR对CPU \#1置0。已经清零了整个set,所以CPU\_CLR清零是多余的,仅仅是处于完整性的考虑。

  在同样的两处理器系统上运行,得到了稍微不同的输出:

\begin{verbatim}
  cpu=0 is set
  cpu=1 is unset
  cpu=2 is unset
  ...
  cpu=1023 is unset
\end{verbatim}

  现在CPU \#1被禁止,该进程无论如何总是运行在CPU \#0上。

  可能的四种错误号如下:

\begin{eqlist*}
\item[EFAULT] 提供的指针在进程的地址空间外或者无效。
\item[EINVAL] 系统中没有处理器被允许调度(仅适用于sched\_setaffinity()),或者setsize小于内核中表示处理器集合的数据结构的大小。
\item[EPERM] pid指明的进程不属于调用进程的有效用户ID,而且该进程没有CAP\_SYS\_NICE能力。
\item[ESRCH] pid指定的进程不存在。
\end{eqlist*}

\section{实时系统}

  在计算机领域,术语“实时”往往是混乱和误解的根源。如果一个系统受到操作期限\footnote[1]{译者注: operational deadlines}---请求和响应之间的最小量和命令次数---的支配,就称该系统是“实时”的。几乎在所有汽车上都可以看到的“防抱死(ABS)“系统就是一个类似的实时系统。在这个系统中,当踩下刹车的时候,计算机通过一秒内多次施加和释放最大刹车压力来调节刹车压力,以防止轮胎“锁死”,保证刹车性能,避免汽车失控。而系统的操作期限就是系统能够多快地响应轮胎“锁死”,能够多快地施加刹车压力。

  包括Linux在内的多数现代操作系统都提供了某些层次的实时支持。

\subsection{软硬实时系统}

  实时系统分为软硬实时系统两大类。硬实时系统对于操作期限要求非常严格,超过期限就会产生失败,后果很严重。另一方面,软实时系统却不认为超过期限是一个严重的失败。

  硬实时系统很容易分辨:防抱死系统、军用武器系统、医疗设备、信号处理都是比较典型的例子。软实时系统则不太容易分辨,一个比较明显的例子是视频处理程序: 如果超过了操作时限,用户会注意到一些质量下降,但是少量的丢帧还是可以忍受的。

  很多应用都有强制的时间要求,如果不能满足要求就会损害用户体验,多媒体应用,游戏和网络程序都在其中。文字编辑器怎么样呢?如果程序不能很快地响应键盘输入,体验就很差,用户会感到愤怒或者有很深的挫败感。它是软实时应用吗?当然,当开发者写程序的时候,他们意识到必须及时地响应输入。但它构成一个强制的操作期限吗?软实时程序的定义说得很清楚。

  和一般看法不同,实时系统并不一定快。实际上,在相同的硬件条件下,实时系统更可能慢于非实时系统,原因在于,即使不考虑其它因素,支持实时进程也会增加系统的负担。类似的,软硬实时系统的区分也不等同于操作时限的长短。在检测到过量的中子流出的几秒钟内,如果SCRAM系统不把控制杆放低一些的话,核反应堆就可能过热。这就是一个有漫长操作期限的硬实时系统。相反地,如果一个视频播放器不能够在100ms内填充回放缓冲区,就会跳过一些帧或者发出结结巴巴的声音,这是一个对操作期限要求较高的软实时系统。

\subsection{延时,抖动和截止期限}

  延时指刺激发生直到响应运行的时间,如果延时小于等于操作期限,系统运行正常。在很多硬实时系统中,操作期限和延时是等价的,系统以固定的间隔在确定时刻处理请求。在软实时系统中,响应不需要那么精确,延时也会出现一些变化,响应只需要在截止期限内发生就行。

  通常很难测量延时,因为测量它必须知道刺激发生的时间。然而,给请求打上时间戳,往往影响及时响应的能力。因此,一般的测量延迟的方法都不这么处理;实际上,人们测量成功响应时间的变化。连续事件中间的时间变化称之为抖动,而不测量延时。

  举例来说,考虑每10ms一次的刺激。为了测量系统性能,我们给响应打上时间戳,确保每10ms响应一次,几次测量之间的偏差就是抖动,我们所测量的就是连续响应间的变化。不知道请求发生的时间,我们也就不知道请求和响应之间的时间差;即使知道请求每10ms一次,我们也不知道第一次请求何时开始。或许更令人惊讶的是,很多测量延时的尝试都搞错了,实际得到了抖动而不是延时。可以肯定的是,抖动是一个有用的测量值,如此的测量也是非常有用的。无论如何,我们必须面对现实!

  硬实时系统经常出现一些非常小的抖动,因为它们在一段时间后而不是那段时间内响应刺激。系统追求零抖动,从而使延时等于操作间隔。如果延时超过间隔,就会失败。

  软实时系统对抖动更宽容。在这些系统中,理想情况下响应时间会在操作间隔内,通常快一些,有时慢一点。因此,抖动就可以完美地替代延时作为性能测量的目标了。

\subsection{Linux的实时支持}

  Linux通过IEEE Std 1003.1b-1993(缩写为POSIX 1993或POSIX.1b)定义的一系列系统调用来为应用程序提供软实时支持。

  技术上讲,POSIX标准并没有说明它提供的实时支持是软还是硬。实际上,POSIX标准仅仅描述了一些基于优先级的调度策略,操作系统服从何种时间约束取决于操作系统设计者。

  过去这些年,Linux内核在不牺牲性能的情况下,取得了越来越好的实时支持,提供越来越小的延时和更加一致的抖动。主要原因在于改进能够帮助很多类应用程序,比如桌面和I/O约束进程,而不仅仅是实时应用;改进对于Linux在嵌入式和实时领域的成功也有很大贡献。

  不幸的是,许多嵌入式和实时领域对Linux内核的修改仅仅存在于定制的Linux解决方案中,独立于主流的官方内核。其中的一些修改进一步减少延时,甚至达到硬实时系统的标准。下面的几节仅仅讨论官方内核接口和主流内核行为。幸运的是,大多数实时修改都利用POSIX接口,因此,接下来的讨论也适用于修改版系统。

\subsection{Linux调度策略和优先级}

  linux对进程的调度行为依赖于进程的调度策略,也称之为调度类别。Linux提供了两类实时调度策略作为正常默认策略的补充。头文件<sched.h>中的预定义宏表示各个策略:分别为SCHED\_FIFO, SCHED\_RR和SCHED\_OTHER。

  每一个进程都有一个与nice值无关的静态优先级,对于普通程序,值为0;对于实时程序,它为1到99。Linux调度器始终选择最高优先级的进程运行(静态优先级数值最大的进程)。如果一个优先级为50的进程在运行,此时一个优先级为51的进程就绪,调度器会直接抢占当前进程,转而运行新到的高优先级进程。相反,如果一个优先级为49的进程就绪,它会一直等待直到优先级为50的进程阻塞才可运行。因为普通进程优先级为0,所以实时进程总会抢占非实时进程开始运行。

\subsubsection{“先进先出”策略}

  先进先出(FIFO)策略是没有时间片的非常简单的实时策略。只要没有高优先级进程就绪,FIFO类型进程就会持续运行,用宏SCHED\_FIFO表示。

  因为缺少时间片,它的操作策略相当简单:

\begin{itemize}
\item \begin{flushleft}一个就绪的FIFO型进程如果是系统中的最高优先级进程就会一直保持运行。特别的,一旦FIFO类进程就绪,它就会直接抢占普通进程。\end{flushleft}
\item \begin{flushleft}FIFO型进程持续运行直到阻塞或者调用sched\_yield(),或者高优先级进程就绪。\end{flushleft}
\item \begin{flushleft}当FIFO型进程阻塞时,调度器将其移出就绪队列。当它恢复时,被插到相同优先级进程队列的末尾。因此,它必须等待高优先级或同等优先级进程停止运行。\end{flushleft}
\item \begin{flushleft}当FIFO型进程调用sched\_yield()时,调度器将其放到同等优先级队列的末尾,因此,它必须等待其它同等优先级进程停止运行。如果没有其它同等优先级进程,sched\_yield()将没有作用。\end{flushleft}
\item \begin{flushleft}当FIFO型进程被抢占,它在优先级队列中的位置不变。因此,一旦高优先级进程停止运行,被抢占的FIFO型进程就会继续运行。\end{flushleft}
\item \begin{flushleft}当一个进程变为FIFO型或者进程静态优先级改变,它将会被放到相应优先级队列的头部。因此,新来的FIFO型进程能够抢占同等优先级进程。\end{flushleft}
\end{itemize}

  本质上,我们可以认为FIFO型进程只要具有最高优先级,就能一直运行。比较有趣的部分在于同等优先级的FIFO进程之间的关系。

\subsubsection{轮转策略}

  轮转策略类似于FIFO类型,仅仅引入了处理同等优先级进程的附加规则,以SCHED\_RR表示。

  调度器给每一个RR型进程分配一个时间片。当RR型进程耗光时间片时,调度器将其放到所在优先级队列的末尾;通过这种方式,RR型进程间就能轮转调度。如果只有一个进程在给定优先级上,RR型就等同于FIFO型,这种情况下,它耗尽时间片,然后直接继续运行。

  我们可以认为RR型进程等同于FIFO型进程,仅仅是在时间片耗尽的时候停止运行,排到同等优先级队列的末尾。

  选择SCHED\_FIFO或者SCHED\_RR仅仅取决于优先级内部的操作,RR型的时间片仅在相同优先级的进程间相关。FIFO进程会不停运行,RR型进程会在给定优先级进程间调度,但是都不会出现高优先级进程等待低优先级进程的情况。

\subsubsection{普通调度策略}

  SCHED\_OTHER代表标准调度策略,适用于默认的非实时进程。所有这些进程的静态优先级都为0,因此,任意就绪FIFO或RR型进程都会抢占它们。

  调度器利用先前讨论过的nice值来划分普通进程的优先级,静态优先级不受nice值的影响,始终为0。

\subsubsection{批调度策略}

  SCHED\_BATCH是批调度或空闲调度的策略,它在某种程度上是实时调度的对立面:这种类型的进程只在系统中没有其它就绪进程时才会运行,即使那些进程已经耗光时间片。这不同于低优先级进程,在那种情况下,进程最终会在高优先级进程耗光时间片后开始运行。

\subsubsection{设置Linux调度策略}

  进程可以通过sched\_getscheduler()和sched\_setscheduler()来操作Linux调度策略:

\begin{lstlisting}
  #include <sched.h>
  struct sched_param {
    /* ... */
    int sched_priority;
    /* ... */
  };
  int sched_getscheduler (pid_t pid);
  int sched_setscheduler (pid_t pid, int policy, const struct sched_param *sp);
\end{lstlisting}

  对sched\_getscheduler()的成功调用将返回由pid指定进程的调度策略,如果pid为0,则返回调用进程的调度策略。<sched.h>中定义了一个整数表示调度策略:SCHED\_FIFO表示先进先出策略,SCHED\_RR表示轮转策略,SCHED\_OTHER表示普通进程。遇到错误,函数返回-1(-1不是有效的调度策略),同时适当地设置错误号。

  用法很简单:

\begin{lstlisting}
  int policy;
  /* get out scheduling policy */
  policy = sched_getscheduler (0);
  switch (policy) {
    case SCHED_OTHER:
      printf ("Policy is normal\n");
      break;
    case SCHED_RR:
      printf ("Policy is round-robin\n");
      break;
    case SCHED_FIFO:
      printf ("Policy is fist-in, first-out\n");
      break;
    case -1:
      perror ("sched_getscheduler");
      break;
    default:
      fprintf (stderr, "Unknown policy!\n");
  }
\end{lstlisting}

  调用sched\_setscheduler()将设置由pid指定进程的调度策略,与策略有关的其它参数则由sp确定。当pid是0时,进程将设置自己的策略和参数。函数成功返回0,失败返回-1并设置错误号。

  sched\_param结构体中的有效字段依赖于操作系统支持的调度策略。SCHED\_RR和SCHED\_FIFO都至少需要一个字段sched\_priority来指明静态优先级。SCHED\_OTHER不使用任何字段,虽然未来的调度策略可能会用到。因此,可移植和合法的程序不应该对结构的布局作出任何假设。

  设置进程调度策略和参数很简单:

\begin{lstlisting}
  struct sched_param sp = { .sched_priority = 1 };
  int ret;
  ret = sched_setscheduler (0, SCHED_RR, &sp);
  if (ret == -1) {
    perror ("sched_setscheduler");
    return 1;
  }
\end{lstlisting}

  代码设置调用进程采用轮转调度策略,优先级为1。我们假设1是有效优先级值---技术上讲,并不必要。我们会在下一节讨论如何得到有效优先级取值范围。

  设置除SCHED\_OTHER外的调度策略都需要CAP\_SYS\_NICE能力,因此,通常由root用户运行实时进程。从2.6.12内核开始,RLIMIT\_RTPRIO资源限制允许非root用户在一定上限内设置实时优先级。

  错误码。错误时设置四种错误值:

\begin{eqlist*}
\item[EFAULT] 指针sp指向的内存区域非法或不可访问。
\item[EINVAL] policy指定的调度策略无效,或者sp值不适用于给定的策略(仅实用于sched\_setscheduler())。
\item[EPERM] 调用进程不具备必要的能力。
\item[ESRCH] pid指定的进程不在运行状态。
\end{eqlist*}

\subsection{设置调度参数}

  POSIX定义的sched\_getparam()和sched\_setparam()接口可以获取和设置已有调度策略的相关参数:

\begin{lstlisting}
  #include <sched.h>
  struct sched_param {
    /* ... */
    int sched_priority;
    /* ... */
  };
  int sched_getparam (pid_t pid, struct sched_param *sp);
  int sched_setparam (pid_t pid, const struct sched_param *sp);
\end{lstlisting}

  sched\_getscheduler()接口仅仅返回调度策略,sched\_getparam()则将pid进程的调度参数存储在sp中:

\begin{lstlisting}
  struct sched_param sp;
  int ret;
  ret = sched_getparam (0, &sp);
  if (ret == -1) {
    perror ("sched_getparam");
    return 1;
  }
  printf ("Our priority is %d\n", sp.sched_priority);
\end{lstlisting}

  pid为0,返回调用进程的参数。成功返回0,错误返回-1,设置错误号。

  因为sched\_setscheduler()也能设置任何调度参数,所以sched\_setparam()仅仅用来稍后修改参数:

\begin{lstlisting}
  struct sched_param sp;
  int ret;
  sp.sched_priority = 1;
  ret = sched_setparam (0, &sp);
  if (ret == -1) {
    perror ("sched_setparam");
    return 1;
  }
\end{lstlisting}

  函数成功则根据sp设置pid指定进程的调度参数,返回0。失败,返回-1,设置errno。

  如果我们顺序运行上文两段代码,应该会看到这样的输出:

\begin{verbatim}
  Our priority is 1
\end{verbatim}

  这个例子再一次假设1是合法值,此时它确实是,但可移植的程序应该主动去确定这一点。稍后我们会看看如何检测有效优先级的范围。

\subsubsection{错误码}

  函数可能设置四种错误码:

\begin{eqlist*}
\item[EFAULT] 指针sp指向的内存区域非法或不可访问。
\item[EINVAL] sp值不适用于给定的策略(仅实用于sched\_getparam())。
\item[EPERM] 调用进程不具备必要的能力。
\item[ESRCH] pid指定的进程不在运行状态。
\end{eqlist*}

\subsubsection{确定有效优先级的范围}

  在上边的例子中,我们把优先级数值硬编码到函数调用中。POSIX并不能保证系统上确定的调度优先级可用,仅仅要求至少有32阶优先级。在“Linux调度策略和优先级”一节中我们曾提到Linux为两类实时调度策略提供了1到99阶共99级。一个清晰可移植的程序通常实现自己的优先级范围,然后映射到操作系统的范围上。比如,你需要四个不同的实时优先级层次,你可以动态地确定优先级范围再从中选择四个。

  Linux提供两个系统调用来获得优先级范围,一个返回最小值,另一个返回最大值:

\begin{lstlisting}
  #include <sched.h>
  int sched_get_priority_min (int policy);
  int sched_get_priority_max (int policy);
\end{lstlisting}

  一旦成功执行,sched\_get\_priority\_min()返回最小值,sched\_get\_priority\_max()返回policy所关联策略的最大有效优先级。两个函数调用成功都返回0(此处为原文错误,正常应该返回对应的值),调用失败返回-1, 唯一可能的错误是policy值非法,此时错误号被设置为 EINVAL。

  用法很简单:

\begin{lstlisting}
  int min, max;
  min = sched_get_priority_min (SCHED_RR);
  if (min == -1) {
    perror ("sched_get_priority_min");
    return 1;
  }
  max = sched_get_priority_max (SCHED_RR);
  if (max == -1) {
    perror ("sched_get_priority_max");
    return 1;
  }
  printf ("SCHED_RR priority range is %d - %d\n", min, max);
\end{lstlisting}

  在标准Linux系统上,代码运行得到:

\begin{verbatim}
  SCHED_RR priority range is 1 - 99
\end{verbatim}

  此前讨论过,数值越大就意味着越高的优先级。下边的代码可以设置进程的相应策略的最高优先级:

\begin{lstlisting}
  /*
  * set_highest_priority – set the associated pid's scheduling
  * priority to the highest value allowed by its current
  * scheduling policy. If pid is zero, sets the current
  * process's priority.
  *
  * Returns zero on success.
  */
  int set_highest_priority (pid_t pid)
  {
    struct sched_param sp;
    int policy, max, ret;
    policy = sched_getscheduler (pid);
    if (policy == -1)
      return -1;
    max = sched_get_priority_max (policy);
    if (max == -1)
      return -1;
    memset (&sp, 0, sizeof (struct sched_param));
    sp.sched_priority = max;
    ret = sched_setparam (pid, &sp);
    return ret;
  }
\end{lstlisting}

  程序一般都是获得系统的最小或最大值,然后按1递增(比如max-1, max-2等等),分配给指定的进程。

\subsection{sched\_rr\_get\_interval()}

  前边说过,SCHED\_RR进程除了拥有时间片外和SCHED\_FIFO进程相同。当SCHED\_RR进程耗光时间片的时候,调度器将其放到同一优先级队列的末尾。通过这种方式,所有相同优先级的SCHED\_RR进程循环运行。无论运行进程是否还有时间片,高优先级进程(包括同等或较高优先级的SCHED\_FIFO进程)总是会抢占它。

  POSIX定义了获得进程时间片长度的接口:

\begin{lstlisting}
  #include <sched.h>
  struct timespec {
    time_t tv_sec;    /* seconds */
    long   tv_nsec;   /* nanoseconds */
  };
  int sched_rr_get_interval (pid_t pid, struct timespec *tp);
\end{lstlisting}

  sched\_rr\_get\_interval()这个糟糕命名的函数,对它的成功调用将把pid指定进程的时间片存储在tp指向的timespec结构中,然后返回0;失败,函数返回-1,设置errno。

  POSIX规定这个函数只能工作于SCHED\_RR进程,然而在Linux上它可以获得任意进程的时间片长度。可移植应用必须假定函数仅能工作于轮转策略,面向Linux的程序可以作一些必要的拓展。例子如下:

\begin{lstlisting}
  struct timespec tp;
  int ret;
  /* get the current task's timeslice length */
  ret = sched_rr_get_interval (0, &tp);
  if (ret == -1) {
    perror ("sched_rr_get_interval");
    return 1;
  }
  /* convert the seconds and nanoseconds to milliseconds */
  printf ("Our time quantum is %.2lf milliseconds\n",
  (tp.tv_sec * 1000.0f) + (tp.tv_nsec / 1000000.0f));
\end{lstlisting}

  如果进程是FIFO类型,则tv\_sec和tv\_nsec都是0,意味着无限长的时间片。

\subsubsection{错误码}

  共有三种可能的错误码:

\begin{eqlist*}
\item[EFAULT] 指针tp指向的内存无效或者不可访问。
\item[EINVAL] pid无效(比如pid是负值)。
\item[ESRCH] pid有效,但指向不存在的进程。
\end{eqlist*}

\subsection{关于实时进程的一些提醒}

  因为实时进程的本质,开发者在开发和调试此类程序的时候应该特别慎重。如果一个实时程序仓促行事,系统可能失去响应。只要没有高优先级程序,实时程序中任何处理器约束的循环---或者是任何不会阻塞的代码---都会永远地运行下去。

  因此,设计实时程序需要小心和注意,这些有很高权限的程序很容易就使系统崩溃。下面是一些技巧和提醒:

\begin{itemize}
\item 切记任何处理器约束的循环,如果没有中断或者高优先级进程来打断它,在完成前都会一直运行。如果是无限循环,系统将失去响应。
\item 因为实时进程运行会占有系统的所有资源,所以在设计的时候需要特别注意,不要过度损害系统其它部分的处理器时间。
\item 小心忙等待。如果一个实时进程忙等待一个较低优先级进程占有的资源,该进程会永远处于忙等待状态。
\item 当开发实时程序的时候,永远开着一个终端,运行一个更高优先级的进程。在紧急情况下,终端会保持响应,允许你杀死失控的进程。(当终端空闲等待输入的时候,不会影响其它实时进程。)
\item util-linux工具包中的chrt实用程序可以使获取和设置实时进程属性更轻松。它可以简单地让任意程序运行在实时调度策略下,比如前述的终端,或者改变已有应用的实时优先级。
\end{itemize}

\subsection{确定性}

  实时进程乐于看到确定性。在实时计算中,如果给予相同的输入,一个动作总是在相同的时间内产生相同的结果,我们就说这个动作是确定的。现代计算机可以说是不确定的集合体:多级缓存(命中与否不可预测),多处理器,分页,交换,和多任务都使估计一个动作需要多长时间变得不可能。显然我们现在每一个动作(相对于硬盘访问)都是“不可思议的快”,但是同时现代系统也使得我们难于精确测量每一个动作的时间。

  实时应用一般会尽量限制不可预测性,和最坏情况下的延时。下面的章节讨论达到目标的两种方法。

\subsubsection{数据故障预测和内存锁定}

  想象一下:ICBM(译者注:洲际导弹系统)监视器接入产生硬件中断,设备驱动迅速地拷贝硬件数据到内核。驱动会注意到一个进程因等待数据阻塞在硬件设备上,因此会通知内核唤醒进程。内核注意到该进程是实时进程且拥有高优先级,就会直接抢占当前运行进程,将注意力转移过来,决定直接调度实时进程。调度器转换到运行实时进程,上下文切换到相应的地址空间。进程开始运行,整个过程需要0.3ms,在1ms的容忍限度内。

  现在考虑用户空间,实时进程注意到接入的ICBM,开始处理轨道。计算好弹道后,实时进程开始配置反导系统。仅仅花掉0.1ms,足够ABM响应和拯救生命。但是---不---ABM的代码已经被交换到硬盘上。于是发生页错误,处理器切回内核模式,启动硬盘I/O来获得交换出去的数据。实时进程会一直休眠直到处理完页错误,几秒钟过去了,一切太晚了。

  显然,分页和交换给实时进程带来了很多不确定性。为了阻止这种灾难,实时应用往往“通过锁定”或者“硬连接“来将地址空间中的页提前放入物理内存,阻止其被交换出去。一旦页被锁定,内核就不会将其交换出去,任何访问都不会引起页错误,大多数实时应用都锁定部分和全部页面到物理内存。

  Linux为两种方法都提供了接口。Chapter 4讨论了故障预测的接口,Chapter 8将讨论锁定内存的接口。

\subsubsection{CPU亲和度和实时进程}

  实时应用的第二个难点在于多任务。虽然Linux内核是抢占式的,但是调度器并不总能直接调度另一个进程。有时,当前进程运行在内核中的临界区,调度器就必须等待它退出临界区,如果此时有一个实时进程要运行,延时将不可接受,很快就会超出操作期限。

  因此,多任务和分页一样也带来了相似的不确定。对于多任务的解决方案也一样:消除它。当然,前提是你不能简单地消灭其它所有进程,如果你可以,你就不需要Linux了---简单定制的操作系统就能满足要求。如果你的系统中有多个处理器,可以指定一个或多个专门用于实时进程。从实际效果上讲,你把实时进程和多任务分离开来。

  本章已经讨论过操作进程CPU亲和度的系统调用。一个潜在的对实时进程的优化是为每一个实时进程保留一个处理器,剩下的处理器由其它进程共享。

  最简单的方法是修改Linux的init程序,SysVinit\footnote[1]{SysVinit源代码可以在ftp://ftp.cistron.nl/pub/people/miquels/sysvinit/取得,以GNU General Public License v2授权},从而可以在启动前做类似下面的动作:

\begin{lstlisting}
  cpu_set_t set;
  int ret;
  CPU_ZERO (&set); /* clear all CPUs */
  ret = sched_getaffinity (0, sizeof (cpu_set_t), &set);
  if (ret == -1) {
    perror ("sched_getaffinity");
    return 1;
  }
  CPU_CLR (1, &set); /* forbid CPU #1 */
  ret = sched_setaffinity (0, sizeof (cpu_set_t), &set);
  if (ret == -1) {
    perror ("sched_setaffinity");
    return 1;
  }
\end{lstlisting}

  代码首先得到初始的可用处理器集合,我们期望是全体处理器。然后移出一个处理器CPU \#1,更新可用处理器集合。

  因为可用处理器集合在父子进程间继承,init又是所有进程的祖先,所以所有的进程都会根据修改后的处理器集合运行,CPU \#1上将没有任何进程。

  接下来,修改实时程序让它仅在CPU \#1上运行:

\begin{lstlisting}
  cpu_set_t set;
  int ret;
  CPU_ZERO (&set); /* clear all CPUs */
  CPU_SET (1, &set); /* allow CPU #1 */
  ret = sched_setaffinity (0, sizeof (cpu_set_t), &set);
  if (ret == -1) {
    perror ("sched_setaffinity");
    return 1;
  }
\end{lstlisting}

  如此,结果就是实时进程仅运行在CPU \#1上,其它的所有进程分享剩下的处理器。

\section{资源限制}

  Linux内核有对进程的资源限制,明确规定了进程可以消耗的内核资源的上限,比如打开文件的数目,内存页数,未处理的信号等等。限制是强制性的,内核不会允许进程的超过这一硬性限制。比如,如果一个打开文件的操作会使得进程拥有的文件超出资源限制,open()调用会失败\footnote[2]{此时,调用会设置错误号为EMFILE,表明进程达到文件数量上限。Chapter 2讨论了open()系统调用}。

  Linux提供了两个操作资源限制的系统调用。两个都是POSIX的标准调用,Linux做了一些补充,分别用getlimit()和setlimit()获取和设置限制:

\begin{lstlisting}
  #include <sys/time.h>
  #include <sys/resource.h>
  struct rlimit {
    rlim_t rlim_cur; /* soft limit */
    rlim_t rlim_max; /* hard limit */
  };
  int getrlimit (int resource, struct rlimit *rlim);
  int setrlimit (int resource, const struct rlimit *rlim);
\end{lstlisting}

  一般用类似RLIMIT\_CPU的整数常量表示资源,rlimit结构表示实际限制。结构定义了两个上限:软限制和硬限制。内核对进程强制施行软限制,但进程自身可以修改软限制,可以是0到硬限制之间的任意值。不具备CAP\_SYS\_RESOURCE能力的进程(比如,非root进程),只能调低硬限制。非特权程序不能提升硬限制,包括恢复为之前的较高的值;因此,调低硬限制是不可逆的。特权进程则可以设置硬限制为任意合法值。

  限制的含义与资源相关。比如资源是RLIMIT\_FSIZE,就表示一个进程可以创建的最大文件长度,单位是字节。此时如果rlim\_cur是1024,则进程不可以创建大于1K的文件,也不能扩展文件至1k以上。

  所有的资源限制都有两个特殊值: 0和无限。前者表示禁止使用资源,例如RLIMIT\_CORE是0,则内核不会创建内存转储文件。相反,后者表示不存在对资源的限制。内核用特殊值RLIM\_INFINITY表示无限,碰巧是-1,(可能和函数调用错误返回-1相混淆)。如果RLIMIT\_CORE是无限,则内核可以创建任意大小的内存转储文件。

  函数getrlimit()会在指针rlim指向的结构中放入resource的软硬限制。成功,返回0;错误,返回-1,设置错误号。

  相对的,函数setrlimit()按rlim指定的值设置resource的软硬限制。成功,返回0,内核更新对应的资源限制;失败,返回-1,设置错误号。

\subsection{限制列表}

  目前Linux提供了15种资源限制:


\begin{itemize}{}
\item RLIMIT\_AS\\
  进程地址空间上限,单位是字节。试图增加地址空间大小超过限制---比如调用mmap()和brk()函数---都会失败,返回ENOMEM。如果进程的栈自动增加,超过了限制,内核将给进程SIGSEGV信号。该限制的值通常为RLIM\_INFINITY。
\item RLIMIT\_CORE\\
  内存转储文件大小的最大值,单位是字节。如果非0,超出限制的内存转储文件将截短为最大限制大小;如果是0,将不会产生此类文件。
\item RLIMIT\_CPU\\
  一个进程可以使用的最长CPU时间,单位是秒。如果进程运行时间超出限制,将接受和处理内核发出的SIGXCPU信号。一个可移植程序必须在接收到该信号时中断,当POSIX没有定义内核的下一步动作。一些系统会中断进程,然而Linux允许进程继续运行,并且每秒发送给该进程一个SIGXCPU信号。一旦进程达到硬限制,将会收到SIGKILL信号并被中断。
\item RLIMIT\_DATA\\
  进程数据段和堆的大小,单位是字节。试图通过brk()来扩大数据段以超出限制将失败并返回ENOMEM。
\item RLIMIT\_FSIZE\\
  文件可以创建的最大文件,单位是字节。如果进程扩展文件超出了限制,内核将发送SIGXFSZ信号。默认情况下,信号将终止进程。但是,进程也可以在系统调用失败返回EFBIG时选择自己捕捉和处理信号。
\item RLIMIT\_LOCKS\\
  进程可以拥有的文件锁的最大数量(参见第七章关于文件锁的讨论)。一旦达到最大值,任何试图获取额外锁的努力都会失败,返回ENOLCK。Linux 2.4.25内核移除了这一功能,在当前内核中,可以设定这一限制,但不会起任何作用。
\item RLIMIT\_MEMLOCK\\
  不具有CAP\_SYS\_IPC能力的进程(非root进程)通过mlock(),mlockall()或者shmctl()能锁定的最多内存的字节数。当超过限制的时候,调用失败范围EPERM。实际上,实际限制向下舍入到整数个内存页。拥有CAP\_SYS\_IPC能力的进程可以锁定任意数量的内存页,此限制不再有效。在2.6.9内核前,该限制作由于有CAP\_SYS\_IPC能力的进程,非特权进程根本不能锁定内存页。此限制不属于POSIX标准,BSD首先引入了它。
\item RLIMIT\_MSGQUEUE\\
  用户可以在POSIX消息队列中分配的最多字节。如果新建的消息超出限制,mp\_open()函数失败返回ENOMEM。它不属于POSIX标准,于2.6.8内核中引入并为Linux独有。
\item RLIMIT\_NICE\\
  进程可以降底nice值(提升优先级)的最大值。本章前文已说明,普通进程只能提高友好度(降低优先级)。这个限制允许管理员规定进程可以合法地提升优先级的级数。因为nice值可能是负值,内核用$20-rlim\_cur$表示。因此,如果限制设置为40,进程友好度最低为-20(最高优先级)。2.6.12内核引入了这个限制。
\item RLIMIT\_NOFILE\\
  该值比进程可以打开的最多文件数大一。任何超出限制的企图都会失败返回EMFILE。在BSD中此限制名字为RLIMIT\_OFILE。
\item RLIMIT\_NPROC\\
  系统任意时刻允许的最多进程数。任何超出限制的企图都会失败,fork()返回EAGAIN。此限制不属于POSIX,由BSD引入。
\item RLIMIT\_RSS\\
  进程可以驻留在内存中的最多页数(即驻留集大小RSS)。仅在早期的2.4内核中强制执行;当前内核允许设置,但不强制执行。此限制不属于POSIX,由BSD引入。
\item RLIMIT\_RTPRIO\\
  没有CAP\_SYS\_NICE能力的进程可以拥有的最大实时优先级。通常,非特权进程不会要求实时调度。此限制不属于POSIX,由2.6.12内核引入并为Linux独有。
\item RLIMIT\_SIGPENDING\\
  用户消息队列中最多信号数。请求更多的信号将失败,sigqueue()这样的系统调用将返回EAGAIN。注意,可以无视这个限制而将一个尚未排入队列的信号实例加入该队列. 因此,总是可以向进程传递SIGKILL和SIGTERM信号。此限制不属于POSIX,由Linux独有。
\item RLIMIT\_STACK\\
  栈的最大字节长度。超出限制将收到SIGSEGV信号。
\end{itemize}

% 根据第一版的勘误,该段被全部替换。
%内核以用户为存储资源限制的基本单位,也就是说,相同用户的所有进程都会有相同的软硬资源限制,但是资源限制本身却描述了对单一进程的限制。比如,内核为每个用户都维护一个RLIMIT\_NOFILE,默认是1024,这个限制规定了每个进程能打开的最多文件数,而不是该用户总共能打开的文件数。还需要注意的是,这并不意味着限制能针对进程分别设置---一旦进程修改了RLIMIT\_NOFILE软限制,此用户的所有进程都会受到这种限制。

% 第二版中的该段翻译如下
内核中以进程为单位管理资源限制。子进程在fork的时候从父进程继承资源限制,限制通过exec进行维护。

\subsubsection{默认限制}

  默认限制取决于三个变量:初始软限制,初始硬限制和系统管理员。内核说明了初始软限制和初始硬限制,如表。内核在init进程中设置这些限制,并通过父子进程间的继承传递给所有的子孙进程。
\begin{center}
\begin{tabular}{ccc}
  \toprule [1pt]
  \rowcolor[gray]{.9}
    Resource limit & Soft limit & Hard limit\\
  \midrule
    RLIMIT\_AS & RLIM\_INFINITY & RLIM\_INFINITY\\
    RLIMIT\_CORE & 0 & RLIM\_INFINITY\\
    RLIMIT\_CPU & RLIM\_INFINITY & RLIM\_INFINITY\\
    RLIMIT\_DATA & RLIM\_INFINITY & RLIM\_INFINITY\\
    RLIMIT\_FSIZE & RLIM\_INFINITY & RLIM\_INFINITY\\
    RLIMIT\_LOCKS & RLIM\_INFINITY & RLIM\_INFINITY\\
    RLIMIT\_MEMLOCK & 8 pages & 8 pages\\
    RLIMIT\_MSGQUEUE & 800 KB & 800KB\\
    RLIMIT\_NICE & 0 & 0\\
    RLIMIT\_NOFILE & 1024 & 1024\\
    RLIMIT\_NPROC & 0 (implies no limit) & 0 (implies no limit)\\
    RLIMIT\_RSS & RLIM\_INFINITY & RLIM\_INFINITY\\
    RLIMIT\_RTPRIO & 0 & 0\\
    RLIMIT\_SIGPENDING & 0 & 0\\
    RLIMIT\_STACK & 8 MB & RLIM\_INFINITY\\
  \bottomrule[1pt]
\end{tabular}
\end{center}

  两种情况可以改变默认限制:

\begin{itemize}
\item 任何进程都可以在0到硬限制的范围内增加软限制,也可以减少硬限制,子进程可以通过fork来继承这些改变。
\item 特权进程可以任意设置硬限制,子进程同样可以通过fork来继承这些改变。
\end{itemize}

  普通进程继承体系中的root进程不太可能修改任何硬限制,因此,第一点更有可能是限制变化的原因。实际上,对于进程的限制往往由用户通过shell设定,系统管理员可以进行设置来提供种类繁多的限制。比如在Bourne-again shell(bash)中,管理员可以使用ulimit命令来设置。注意管理员不仅可以降低限制值,还可以提升软限制到硬限制,从而给用户提供更合理的限制。一般会使用RLIMIT\_STACK(在许多系统上该值都被设置为RLIM\_INFINITY)进行处理。

\subsection{获取和设置资源限制}

  解释了种种资源限制,让我们来考察一下获取和设置这些限制。获取限制很简单:

\begin{lstlisting}
  struct rlimit rlim;
  int ret;
  /* get the limit on core sizes */
  ret = getrlimit (RLIMIT_CORE, &rlim);
  if (ret == -1) {
    perror ("getrlimit");
    return 1;
  }
  printf ("RLIMIT_CORE limits: soft=%ld hard=%ld\n",
  rlim.rlim_cur, rlim.rlim_max);
\end{lstlisting}

  编译然后运行代码将得到:

\begin{verbatim}
  RLIMIT_CORE limits: soft=0 hard=-1
\end{verbatim}

  可以看到,软限制为0,硬限制为-1(-1代表无限)。因此我们可以设置软连接为任意值。下面的例子设置内存转储文件最大为32MB:
\begin{lstlisting}
  struct rlimit rlim;
  int ret;
  rlim.rlim_cur = 32 * 1024 * 1024; /* 32 MB */
  rlim.rlim_max = RLIM_INFINITY; /* leave it alone */
  ret = setrlimit (RLIMIT_CORE, &rlim);
  if (ret == -1) {
    perror ("setrlimit");
    return 1;
  }
\end{lstlisting}

\subsubsection{错误码}

  可能有三种错误代码:

\begin{eqlist*}
\item[EFAULT] rlim指向的内存非法或不可访问。
\item[EINVAL] resource的值非法,或者rlim.rlim\_cur的值大于rlim.rlim\_max(仅适用于setrlimit())。
\item[EPERM] 调用者没有CAP\_SYS\_RESOURCE能力却试图提升硬限制。
\end{eqlist*}
